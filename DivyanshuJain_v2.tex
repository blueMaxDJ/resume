\makeatletter
\newcommand{\rmnum}[1]{\romannumeral #1}
\newcommand{\Rmnum}[1]{\expandafter\@slowromancap\romannumeral #1@}
\makeatother

\documentclass[11pt,a4paper]{moderncv}

% moderncv themes
\moderncvtheme[grey]{banking}                 % optional argument are 'blue' (default), 'orange', 'red', 'green', 'grey' and 'roman' (for roman fonts, instead of sans serif fonts)
%\moderncvtheme[green]{banking}                % idem

% character encoding
\usepackage[utf8]{inputenc}                   % replace by the encoding you are using

% adjust the page margins
\usepackage[scale=0.94]{geometry}
%\setlength{\hintscolumnwidth}{3cm}						% if you want to change the width of the column with the dates
%\AtBeginDocument{\setlength{\maketitlenamewidth}{6cm}}  % only for the classic theme, if you want to change the width of your name placeholder (to leave more space for your address details
%\AtBeginDocument{\recomputelengths}                     % required when changes are made to page layout lengths

% personal data
\firstname{Divyanshu}
\familyname{Jain}
% \title{Divyanshu Jain}               % optional, remove the line if not wanted
%% \address{525 Victoria Street, Costa Mesa}{California}    % optional, remove the line if not wanted
\mobile{(310) 500 9665}                    % optional, remove the line if not wanted
%% \phone{phone (optional)}                      % optional, remove the line if not wanted
%% \fax{fax (optional)}                          % optional, remove the line if not wanted
\email{divyanshu.jain@gmail.com}                      % optional, remove the line if not wanted
\homepage{https://www.linkedin.com/in/divyanshu-jain-b539b5a/}                % optional, remove the line if not wanted
%% \extrainfo{additional information (optional)} % optional, remove the line if not wanted
%\photo[64pt]{mugshot}                         % '64pt' is the height the picture must be resized to and 'picture' is the name of the picture file; optional, remove the line if not wanted
%% \quote{Improve users' lives}                 % optional, remove the line if not wanted

% to show numerical labels in the bibliography; only useful if you make citations in your resume
\makeatletter
\renewcommand*{\bibliographyitemlabel}{\@biblabel{\arabic{enumiv}}}
\makeatother

% bibliography with mutiple entries
%\usepackage{multibib}
%\newcites{misc,misc}{{Others},{Others}}

\nopagenumbers{}                             % uncomment to suppress automatic page numbering for CVs longer than one page
%----------------------------------------------------------------------------------
%            content
%----------------------------------------------------------------------------------
\begin{document}
\maketitle
\vspace*{-10mm}
\section{Industry Experience}
\cventry{2016--present}{System Design Engineer}{NXP Semiconductors }{Irvine}{}{ 
\begin{itemize}%
\item Designed,simulated, coded, tested and verified parts of NXP's first UWB chip targetting secure indoor localization.
\item Tested and verified reader demodulation design of the latest NFC chip using system Verilog.
\item Created modem design verification infrastructure using python MATLAB and bash.
\end{itemize}}
\cvitem{Design Process}{Algorithm exploration in Simulink/MATLAB $\Rightarrow$ Auto-generation of RTL and C from Simulink $\Rightarrow$  Performance Verification in C.}

\cventry{2013--2016}{System Design Engineer}{Broadcom Corp.}{Irvine}{}{ %
\begin{itemize}%
\item Developed part of ultra low power Bluetooth and ZigBee receiver. Algorithmic and systemic approaches were employed to achieve the objective.
\item Designed a template for converting C++ algorithms to RTL using HLS tools like FORTE.
\item Verified  frequency synchronization algorithms for Broadcom's NFC chips. Also, hacked other competitor's NFC chips to compare and understand the performance of their receiver's algorithm
\item Wrote template code for converting Simulink designs to C++ using Real Time Workshop.
\item Created scalable infrastructure in python, MATLAB to perform functional verification, bitmatching and code coverage to validate the design.
\end{itemize}}
\cvline{Design Process}{Algorithm exploration in Simulink/MATLAB/C; Performance evaluation in hand-written C; RTL auto-generation from Forte/Simulink, Bitmatching between C and RTL.}


\cventry{2008--2013}{Member of Technical Staff}{Mojix Inc.}{Los Angeles}{}{ %Firmware Development for RFID Readers
\begin{itemize}%
\item Developed firmware and hardware for Mojix RFID Interrogators (implement real time protocol standardized by EPC to query and manage tag population). 
\item Wrote microcontroller (ARM Cortex-M3) code in C for Mojix low power signal distributors and regenerators. 
\item Implemented a conductive testing procedure to characterize performance of RFID Readers via BER curves by generating random tag signal mixed with Gaussian noise at different Eb/No values.
\item Member of the team involved in designing and implementing a proprietary protocol to transmit protocol commands and receive sensor data, tag data over a wired link passing through multiple devices in a massive distributed network.
\item Coded and developed a part of the  python - Tk based GUI to control, test and setup parameters of our RFID system.
\end{itemize}}
\cvline{Design Process}{Algorithm exploration and performance evaluation in handwritten C; Bitmatching between handwritten C and RTL. }


\cventry{2008--2008}{Interim Engineering Intern}{Qualcomm Inc.}{San Diego}{}{ %Improved Google's Payment Fraud detection system
\begin{itemize}%
\item Wrote a high level design document for carrier frequency offset estimation.
\item Visualized the LTE modem system timeline using GTK wave analyzer.  
\end{itemize}}

\section{Technical skills}
\cvcomputer{Languages}{C/C++, Assembly, Verilog, Python, Java, Tcl}{Dev Tools}{MATLAB, Simulink, Verdi, FORTE, IDEs}
\cvcomputer{Platform}{Linux, Windows }{Software}{Office, Version Control Systems }
\cvcomputer{Concepts}{Signal Processing,  Filters, Probability,\newline Artificial Intelligence, Algorithms}{Hardware}{Pattern and Waveform Generators,\newline Oscilloscopes, Spectrum Analyzers}

\section{Education}
\cventry{2006--2007}{Masters in Electrical Engineering, GPA 3.7/4.0}{University of California}{Los Angeles, California}{}{}

\cvline{Masters Thesis}{\emph{Optimization techniques for Implementing Real Time MIMO channel estimation on a DSP (TI-C64x)}}
\begin{itemize}%
\item Simulated minimum number of channel coefficient bits required to be within permissible error bounds.
\item Optimized hand written assembly code via novel techniques:
	\begin{itemize}%
	\item Leveraged input data properties. (Used simulated number of bits for division algorithm selection)
	\item Equalized load across different functional units of the processor. (by using suboptimal instructions)
	\item Eliminated cross path stalling by manual scheduling.
	\end{itemize}
%\item Further optimized the assembly using conventional techniques of loop unrolling and software pipelining.
\end{itemize}

\cventry{2002--2006}{Bachelors in Electrical Engineering, Aggregate 80.2/100.0}{Madhav Institute of Technology and Science}{Gwalior, India}{}{}


\section{Awards}
\cventry{June 2005}{Bhojwani scholarship for securing highest marks in junior year.}{Academic Distinction}{M.I.T.S. Gwalior, India}{}{}

\section{Patents}
\begin{itemize}%
\item Dong-U Lee, Divyanshu Jain,  2016.{\emph{Feedback-based adaptive load modulation (ALM) for a near field communication (NFC) device}}, U.S. Patent \textbf{US9590701 B2} filed July 27, 2015, and issued Mar 7, 2017.
\item Manolis Frantzeskakis, Dong-U Lee, Divyanshu Jain, Jianhua Gan, Shengyang Xu, 2016. {\emph{Carrier synchronization appropriate for alm nfc data transmission}} U.S. Patent \textbf{US20160241384 A1} filed October 20, 2015, and issued August 18, 2016.
\end{itemize}

\end{document}


%% end of file `template_en.tex'.
